\begin{abstract}
%\boldmath
Minions ipsum me want bananaaa! Hahaha chasy hana dul sae belloo! Aaaaaah daa ti aamoo! La bodaaa pepete. La bodaaa para tú gelatooo underweaaar po kass tank yuuu! Underweaaar. Wiiiii bappleees jiji chasy underweaaar. Tank yuuu! jeje me want bananaaa! Me want bananaaa! Hahaha aaaaaah chasy hana dul sae gelatooo jeje tulaliloo. Baboiii bappleees wiiiii daa underweaaar tank yuuu! Underweaaar. Jiji potatoooo butt la bodaaa ti aamoo! Para tú tatata bala tu. Jiji underweaaar bappleees belloo! La bodaaa. Bee do bee do bee do jiji tatata bala tu belloo! Gelatooo bappleees hana dul sae ti aamoo!

Poopayee bappleees underweaaar bee do bee do bee do. Hana dul sae potatoooo bappleees la bodaaa baboiii. Pepete para tú hana dul sae tank yuuu! Jiji poulet tikka masala tatata bala tu ti aamoo! Po kass belloo! Wiiiii daa me want bananaaa! Chasy poopayee jeje po kass.
\todo[inline]{Change me}
\end{abstract}
% IEEEtran.cls defaults to using nonbold math in the Abstract.
% This preserves the distinction between vectors and scalars. However,
% if the journal you are submitting to favors bold math in the abstract,
% then you can use LaTeX's standard command \boldmath at the very start
% of the abstract to achieve this. Many IEEE journals frown on math
% in the abstract anyway.

% Note that keywords are not normally used for peerreview papers.
\begin{IEEEkeywords}
Android,  Application Architecture, Model-View-Presenter (MVP), App Prototype.
\end{IEEEkeywords}
